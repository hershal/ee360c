\documentclass[12pt]{book}
\title{The EE360C Manual}
\author{Hershal Bhave (hb6279)}
\date{Updated \today}

\usepackage{multicol}
\usepackage[in]{fullpage}
\usepackage{xcolor}
\usepackage{rotating}
\usepackage{mathtools}
\usepackage{amssymb}
\usepackage{cleveref}
\usepackage[nosolutionfiles]{answers}
\usepackage[acronym]{glossaries}

\newenvironment{Ex}{\vspace{1.5em}\hspace{-1.5em}\textbf{Problem}\\}{}
\Newassociation{solution}{Soln}{Answers}
\pagebreak[3]
\newcommand{\Opentesthook}[2]{\Writetofile{#1}{\protect\section{#1: #2}}}
\renewcommand{\Solnlabel}[1]{\textbf{Solution}\quad}

\newcommand{\dd}[1]{\:\mathrm{d}{#1}}
\newcommand{\ddt}[1]{\frac{\dd{}}{\dd{#1}}}
\newcommand{\dddt}[1]{\frac{\dd{}^2}{\dd{#1}^2}}

\makeglossaries
\begin{document}
\maketitle
\tableofcontents
\chapter{Introduction}
\section{What is an Algorithm?}
An algorithm is a set of steps which guarantees a solution to a
problem. It must guarantee to generate a solution and complete in a
reasonably finite amount of time.

\section{What are aome applications of algorithms?}
\begin{multicols}{4}
  \begin{itemize}
  \item FFT
  \item Bitcoins
  \item ray-tracing
  \item path-finding
  \end{itemize}
\end{multicols}
\section{How important are they?}
\subsection{Algorithms vs. Hardware}

\section{Beyond Technology}
Algorithms go beyond technology to influence our lives in novel and
interesting ways.

\chapter{Review of Discrete Math}
\section{Set Definitions}
A set is a collection of {\em distinguishable} objects, called members
of elements. If $x$ is an element of a set $S$, we write $x \in S$. If
$x$ is not an element of a set $S$, we write $x \not\in S$. 

\subsection{Set Operators}
\begin{description}
\item[subset] if $x \in A$ implies $x \in B$, then $A \subseteq B$
\item[proper subset] etc
\end{description}

\section{Relation Definitions}
A binary relation $R$ on two sets $A$ and $B$ is a subset of the
Cartesian product $A \times B$. If $(a,b) \in R$, we sometimes write
$a\,R\,b$.

\subsection{Reflexive Binary Relations}
$R \subseteq A \times A$ is reflexive if $a\,R\,a$ for all $a \in A$.

\subsection{Symmetric Binary Relations}
$R$ is symmetric if $a\,R\,b$ implives $b\,R\,a$ for all $a,b \in A$.

\subsection{Transitive Binary Relations}
$R$ is transitive if $a\,R\,b$ and $b\,R\,C$ imply $a\,R\,c$ for all
$a,b,c \in A$.

\subsection{Antisymmetric Binary Relations}
$R$ is antisymmetric if $a\,R\,b$ and $b\,R\,a$ implies $a = b$.

\subsection{Equivalence Binary Relations}
A relation that is reflexive, symmetric, and transitive is an
equivalence relation. If $R$ is an equivalence relation on set $A$,
then for $a \in A$, the equivalence class of $a$ is the set
$[a] = \{ b \in A : a\,R\,b \}$.

\begin{Ex}
  Consider $R = \{ (a,b) : a, b, \in N \text{ and } a+b \text{ is an
    even number}\}$. Is it reflexive? Is it symmetric? Is it
  transitive?
  \begin{solution}
    
  \end{solution}
\end{Ex}

\subsection{Partial Order}
A relation that is reflexive, antisymmetric, and transitive, is a
partial order.

\section{Function Definitions}
Given two sets $A$ and $B$, a function $f$ is a binary relation on $A
\times B$ such that for all $a \in A$, there exists exactly one $b \in
B$ such that $(a,b) \in f$. The set $A$ is the domain of $f$, where
$a$ is an argument to the function. The set $B$ is the co-domain of
$f$, where $b$ is the value of the function. We often write functions
as the following:
\begin{equation}
  \label{eq:def-ex}
   $f : A \rightarrow B$ \text{\huge SOMETHING}
\end{equation}

\subsection{Finite Sequence}
A finite sequence is a function whose domain is ${0,1,\ldots,n-1}$,
often written as $\langle f(0), f(1), \ldots, f(n-1) \rangle$.

\subsection{Infinite Sequence}
An infinite sequence is a function whose domain is the set of
$\mathbb{N}$ natural numbers ($\{0,1,\ldots\}$).

When the domain of $f$ is a Cartesian product, e.g. $A=A_1 \times A_2
\times \ldots \times A_n$, we write $f(a_1, a_2, \ldots, a_n)$ instead
of $f((a,_1, a_2, \ldots, a_n))$.

We call each $a_i$ an argument of $f$ even though the argument is
really the n-typle $(a_1, a_2, \ldots, a_n)$.

if $f:A \rightarrow B$ is a function and $b = f(a)$, then we say that
$b$ is the image of $a$ under $f$. The range of $f$ is the image of
its domain (i.e., $f(A)$). A function is a surjection of its range if
its range is its codomain. This is sometimes referred to as mapping
$A$ onto $B$. For example, $f(n) = \lfloor \frac{n}{2} \rfloor$. is a
surjective functino from $\mathbb{N}$ to $\mathbb{N}$.

\chapter{Alogrithm Analysis}

\chapter{Graphs and Graph Algorithms}

\chapter{Greedy Algorithms}

\chapter{Divide and Conquer}

\chapter{Dynamic Programming}

\chapter{Network Flow}

\chapter{NP-Completeness}

\chapter{}
\printglossaries
\end{document}