\documentclass{article}
\title{EE360C Programming Homework 1}
\author{Hershal Bhave (hb6279)}
\date{Due October 1, 2014}

\begin{document}
\maketitle

\section{Pseudo-Code}
The pseudo-code for finding a MWMCM for a CBG with potentially
negative weights is not as intuitive as one would think; thus a
brute-force approach was used. \verb|vector_of_cbgs| is a list of
CBGs, \verb|tic_nodes| is a vector of \verb|tic_node|, and
\verb|tac_nodes| is a vector of \verb|tac_nodes|. The brute-force
approach I used was to calculate each and every permutation of the
order of nodes on one side of the graph picking the maximum edge
weight for a node on the other side of the graph. The side that is
chosen for this is dependent on which side has more nodes (i.e. if
there are more tics than tacs, then I generate all the possible
permutations of tics visiting tacs, and vice versa if there are more
tacs than tics). I keep track of these permutations by encoding them
in Lehmer Codes, which can uniquly identify a specific permutation by
a simple index. This can be converted back and forth, but since I'm
storing all the Lehmer Codes, I don't have to worry too much about
going from Lehmer Code to index (I compute the other way around). Once
these codes have been computed, I then start connecting nodes. I do
this by going through the Lehmer permutation and selecting the
connecting node with the maximum weight. I add that to my results
package and disable the node on the other side. This makes sure that
no other node can potentially try to access the node I just used. The
results package keeps track of the cardinality and overall weight of
the CBG match. Once all the nodes have been iterated through, I then
add the results package to a sorted map of results, first sorted by
cardinality and then by weight. Once all the major-nodes have been
iterated through, I simply have to inspect the first index of the
generated cardinality-weight-map to get the MWMCM. This is done for
each instance in the test file.

{
  \footnotesize
\begin{verbatim}
auto find_mwmcms() ->
    std::map<size_t, std::map<int32_t, std::set<mwmcm, mwmcm::mwmcm_compare> > > {

        calculate_adjacency_lists();
        generate_lehmers();
        generate_cardinality_weight_map();
}

auto complex_bipartite_graph::calculate_adjacency_lists() -> void {
    /* Calculate the adjacency lists */
    for (auto tic : tic_nodes) {
        for (auto tac : tac_nodes) {
            if (tac->get_id() <= tic->get_max()
                && tac->get_id() >= tic->get_min()) {
                tic->add_adjacent_node(tac);
                tac->add_adjacent_node(tic);
            }
        }
    }
    /* Each node's adjacency list is sorted by weight */
    for (auto tic : tic_nodes) { tic->sort_adjacent_nodes(); }
    for (auto tac : tac_nodes) { tac->sort_adjacent_nodes(); }
}

auto complex_bipartite_graph::generate_lehmers() -> void {

    /* Generate structures here */
    std::cout << "generating data structures\n";

    if (tic_nodes.size() < tac_nodes.size()) {
        tictac_switch = kUseTics;
    } else {
        tictac_switch = kUseTacs;
    }

    for (auto i=0; i<num_lehmers(); ++i) {
        generate_lehmer_for_idx(i);
    }
}

auto complex_bipartite_graph::generate_cardinality_weight_map() -> void {

    for (const auto idxlehmer : index_lehmer_map) {
        const auto idx = idxlehmer.first;
        const auto lehmer = idxlehmer.second;

        reset();

        mwmcm results;

        /* TODO: What if two edges have the same weight?  */
        for (const auto id : lehmer) {

            std::shared_ptr<tac_node> node = 0;

            switch (tictac_switch) {
            case kUseTics: node = get_tic_with_id(id); break;
            case kUseTacs: node = get_tac_with_id(id); break;
            default:
                std::cout << "Please generate lehmers first...\n";
                exit(1);
            }

            /* The vector is sorted */
            for (auto adj_node : node->get_adjacent_nodes()) {
                /* Picked the highest one */
                if (node->is_enabled() && adj_node->node->is_enabled()) {
                    node->disable();
                    adj_node->node->disable();

                    switch (tictac_switch) {
                    case kUseTics:
                        results.add_association(id,
                                                adj_node->node->get_id(),
                                                adj_node->edge_weight);
                        break;
                    case kUseTacs:
                        results.add_association(adj_node->node->get_id(),
                                                id,
                                                adj_node->edge_weight);
                        break;
                    default:
                        std::cout << "Please generate lehmers first...\n";
                        exit(1);
                    }
                }
            }
        }
        /* To enforce sorting */
        cardinality_weight_mwmcm_map[results.get_cardinality()]
            [results.get_weight()].insert(results);
    }
}
\end{verbatim}
}

\section{Runtime Complexity}
Since this algorithm uses a brute force approach, the runtime is $\mathcal{O}(n!)$.

\end{document}