\documentclass{article}
\title{EE360C Programming Homework 3}
\author{Hershal Bhave (hb6279)}
\date{Due December 4, 2014}

\begin{document}
\maketitle

\section{Proof}
The basic problem for this assignment can be modeled through a DFS
search. Each fragment is checked to determine if it could possibly
start the desired string. If the fragment matches the beginning of the
desired string, then matching segment of the desired string is chopped
off, the current fragment is added to already-matching
\verb|std::vector| of fragments, and the function recurses. This is
repeated for each fragment in the input list. If the recursive call
falls through, the fragment which was just added to the matching
fragments is popped off and the function continues to the next
fragment if available. When the desired word becomes an empty string,
then the list of fragments which match the original desired string has
been found, and this result is added to the list of results. Since I
can relate this problem to a DFS search, I can use a relevant proof to
prove its correctness:

Suppose that my DFS does not find the appropriate solution even though
it exists. Since my DFS goes through all possible paths from the
source fragment list to the desired string, it must find an
appropriate path to the desired string. This is a contraction of the
original supposition. Thus my ``poor man's DFS'' must find a solution
if it exists.

\pagebreak[4]
\section{Pseudo-Code}
{\footnotesize
\begin{verbatim}
auto fragment_assembler::begins_with
    (const std::string* desired, const std::string* fragment_string) const
    -> size_t {

    if (desired->find(*fragment_string) == 0) {
        return fragment_string->size();
    } else {
        return 0;
    }
}

auto fragment_assembler::poor_mans_dfs
    (std::string desired_string,
     std::vector<std::string> current_strings,
     std::vector<std::vector<std::string> >* results) 
    -> bool {

    if (desired_string.size() == 0) {
        results->push_back(current_strings);
        return 1;
    }

    for (const auto str : fragments) {
        size_t chop = begins_with(&desired_string, &str);
        if (chop) {
            current_strings.push_back(str);
            std::string substr = desired_string.substr(chop, desired_string.size());
            poor_mans_dfs(substr, current_strings, results);
            current_strings.pop_back();
        }
    }
    return 0;
}
\end{verbatim}
}

\section{Runtime Complexity}
This algorithm does not disable fragments when they are visited, so
its runtime complexity is at maximum $\mathcal{O}(n^n)$. That being
said, the limiting criteria for the fragments for which the function
is allowed to recurse is very narrow, which makes the actual runtime
far better than $\mathcal{O}(n^n)$. This is done by the
\verb|begins_with| function, which trims down the number DFS searches
considerably. Thus, although I define the runtime complexity as
$\mathcal{O}(n^n)$, it almost never is the case.
\end{document}